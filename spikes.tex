%%%%%%%%%%%%%%%%%%%%%%%%%%%%%%%%%%%%%%%%%
% Medium Length Professional CV
% LaTeX Template
% Version 2.0 (8/5/13)
%
% This template has been downloaded from:
% http://www.LaTeXTemplates.com
%
% Original author:
% Trey Hunner (http://www.treyhunner.com/)
%
% Important note:
% This template requires the resume.cls file to be in the same directory as the
% .tex file. The resume.cls file provides the resume style used for structuring the
% document.
%
%%%%%%%%%%%%%%%%%%%%%%%%%%%%%%%%%%%%%%%%%




%----------------------------------------------------------------------------------------
%	PACKAGES AND OTHER DOCUMENT CONFIGURATIONS
%----------------------------------------------------------------------------------------

\documentclass{resume} % Use the custom resume.cls style

\usepackage[left=0.75in,top=0.6in,right=0.75in,bottom=0.6in]{geometry} % Document margins

\name{Wesley Spikes}
\address{3767 Clarington Ave, Apt 320 \\ Los Angeles, CA 90034}
\address{(909)~$\cdot$~815~$\cdot$~1079 \\ wesley.spikes@gmail.com}

\begin{document}

%----------------------------------------------------------------------------------------
%	SUMMARY
%----------------------------------------------------------------------------------------

\begin{rSection}{Summary}
I'm just me. I like challenges. (FIXME)
Seeks challenges and opportunities for growth.
Handles ambiguous situations well, and can reduce them down to actionable tasks.
Can juggle well when there's many things going on.
\end{rSection}

%----------------------------------------------------------------------------------------
%	TECHNICAL STRENGTHS SECTION
%----------------------------------------------------------------------------------------

\begin{rSection}{Technical Strengths}

\begin{tabular}{ @{} >{\bfseries}l @{\hspace{6ex}} l }
Computer Languages & Perl, Bash, Python \\
Operating Systems & Linux (Debian, Ubuntu, Gentoo), OS X, Windows \\
Software & Apache, Nginx, git, vim, MySQL \\
\end{tabular}

\end{rSection}

%----------------------------------------------------------------------------------------
%	WORK EXPERIENCE SECTION
%----------------------------------------------------------------------------------------

\begin{rSection}{Experience}

\begin{rSubsection}{Personal Break}{Sep 2016 - Present}{}{}
\item Took a 6 month break, to travel, recharge, and explore LA. Best decision.
\item Learned some C\# and Unity to mentor my youngest brother, who wants to get into game development.
\item Playing most of the puzzle games in my Steam backlog.
%\item Reverse engineering games for fun. After beating a game, it's a fun challenge to reverse it and find little ways to tweak them. (FIXME: Comments below)
%   Example projects: Cookie Clicker (simple/boring) and AdCap.
%   AdCap had a non-standard LZF compression routine (slightly non-standard/buggy, as other lzf libs wouldn't decompress it), as well as writing a custom MITM server to "validate" support codes to give free resources.
\item Set up a personal lab in my living room with 4 servers.
%   Configured the lab's gateway server as a router w/ DHCP on the lab network
%   Configured TFTP and DHCP server for bootstrapping bare metal machines
\item Reverse engineering hardware/embedded firmware for fun.
%   Goal is wanting to better understand RE'ing embedded systems
%   Example projects: Sonos Play:1 (current) and Berg Little Printer (past)
%   Sonos Play:1 firmware is hampered by not having a license for a good RE tool like IDA -- Hopper, Binary Ninja, and similar affordable tools don't do a good job of processing binaries for this arch (PowerPC Gen2 a la PowerQUICC II Pro). :\
\end{rSubsection}

%------------------------------------------------

\begin{rSubsection}{DreamHost}{Feb 2015 - Aug 2016}{Senior/Lead Software Engineer}{Los Angeles, CA}
\item Managed two direct reports.
%  What about the mentoring I'd done? Worth calling out specifically?
%  One was a Jr Dev, who I'd worked with to get them to leveled up to Dev.
%  The other Jr was on track to level up when I left, and within a few months of my departure, has successfully gone to Dev. (This one probably needs some massaging, since it feels a bit wrong)
%  Beyond simply mentoring in place, encouraging them to reach out to other teams directly and be confidence enough in that regard.
\item Inter-team cooperation with Systems Engineering, Security, Technical Support, and Corporate IT.
\item Managed core development infrastructure (including Gerrit, Jenkins, and other dev hosts)
\item Entrusted with various sets of secure credentials. Primary maintainer of deployed credentials.
\item Post-exploit incident response
\item Modernized how the primary codebase and it's dependencies were deployed (fully documented), cutting deployment time down from 20 minutes to around 45 seconds.
\item Technical lead on projects to integrate the security team's website malware scanner as well the DreamPress 2 and VPS Billing+SSD projects. (FIXME: I don't like the phrasing on this.)
\item Order of Baku - More than 11 significant security vulnerabilities reported. (Stopped counting, lol.)
\end{rSubsection}

%------------------------------------------------

\begin{rSubsection}{DreamHost}{Jan 2011 - Feb 2015}{Software Engineer}{Los Angeles, CA}
\item Configuration management a la Chef
\item Projects: (FIXME: Decide on some projects to highlight from the comments below)
%   Dedi Relaunch
%   ETL and report building for billing systems (chart.io)
%   Replicon integration (maybe. that was a nightmare project)
%   mod_pagespeed launch
%     Google launched it with GoDaddy announced as a launch partner. While Google was still "working with GoDaddy" to roll it out it, our team managed to test it and launch it, beating GoDaddy to market.
%     https://www.dreamhost.com/blog/2010/11/04/mod_pagespeed-now-available/
%   Svn->Git transition (from end-to-end -- docs, support, migration itself, etc)
%   LogicBoxes (maybe. that was a nightmare project)
\item Major contributions to the foundations/core of a old codebase (1997), without breaking production.
%    Proud of this, given the age and the fact that DH doesn't have good testing overall. (FIXME: Phrasing, also see the comments below.)
%   Object->_classdef (deprecating and safely removing v0 and v1 classdefs, with no errors, issues, or breakage)
%   Object->AUTOLOAD (deprecating and safely removing runtime accessor generation from a large amount of the codebase; project was still in progress at the time I left)
%   Nza->AjaxStep (making Ajax support a first-class citizen of the in-house web framework, without breaking existing code -- essentially converting a hacked-in only-sometimes-working bug -- no joke, a bug -- to an actual feature.
%   Nza -- steps can now 'use strict', which wasn't possible before. Code without 'use strict' in Perl land is a pretty massive issue, leading to many a . In the process of migrating existing steps to this new system, we discovered quite a few potential security issues that we were able to fix before they were ever discovered.
%   Controllers / central launcher / servicectl
%   Services -- migrated executable commands to an actual naming prefix to ensure that code isn't inadvertently and inappropriately executed, while still not breaking old code (sc_renamed)
\item Notable contributions to the billing systems, improving it's resilience and extensibility.
\item Throughly documented how the fraud detection systems worked and what it checked for.
%   That billing disaster that took a couple weeks to clean up (INVNUM, or Merchant Invoice Number, being documented as Optional in the PFPro docs, but the library we were using was using it in a rather terribad way.)
\item Researched and documented tons of obscure parameters and features of the systems that we'd depended on without knowing what they actually did or how they worked.
\item Panel Sec Improvements
\item Audits -- for financial, security, admin, business, whatever
\item Introduced and implemented CI processes (from end-to-end -- implementation, buy-in, support, etc etc)
\end{rSubsection}

%------------------------------------------------

\begin{rSubsection}{DreamHost}{Nov 2009 - Jan 2011}{Technical Support}{Los Angeles, CA}
\item In these days, Support was functionally similar to that of Junior Systems Admin.
\item Specialized in tackling problems others avoided, where it was difficult to find the root cause.
% For example, when a customer's databases all disappeared on a routine mysqld restart, successfully locating, restoring, and sending a summary on what went wrong.
\item Diagnosing loady systems and troubleshooting customer websites.
\item Configuring systems - Linux-VServer, DNS, Apache, Nginx, users, and PHP.
\item Helping customers with a 99\% satisfaction rate.
\end{rSubsection}

%------------------------------------------------

\begin{rSubsection}{Page Provider}{Dec 2007 - Nov 2009}{Systems Admin}{Rancho Cucamonga, CA}
\item Small, 1-rack hosting provider.
\item Building and deploying custom kernels and software for colo'd customers.
\item Provisioning and deploying servers and services.
\item Configuration and management of Apache, users and databases.
\end{rSubsection}

\end{rSection}

%----------------------------------------------------------------------------------------

\end{document}
