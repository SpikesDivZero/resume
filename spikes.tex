%%%%%%%%%%%%%%%%%%%%%%%%%%%%%%%%%%%%%%%%%
% Medium Length Professional CV
% LaTeX Template
% Version 2.0 (8/5/13)
%
% This template has been downloaded from:
% http://www.LaTeXTemplates.com
%
% Original author:
% Trey Hunner (http://www.treyhunner.com/)
%
% Important note:
% This template requires the resume.cls file to be in the same directory as the
% .tex file. The resume.cls file provides the resume style used for structuring the
% document.
%
%%%%%%%%%%%%%%%%%%%%%%%%%%%%%%%%%%%%%%%%%




%----------------------------------------------------------------------------------------
%	PACKAGES AND OTHER DOCUMENT CONFIGURATIONS
%----------------------------------------------------------------------------------------

\documentclass{resume} % Use the custom resume.cls style

\usepackage[left=0.75in,top=0.6in,right=0.75in,bottom=0.6in]{geometry} % Document margins

\name{Wesley Spikes}
\address{3767 Clarington Ave, Apt 320 \\ Los Angeles, CA 90034}
\address{909.815.1079 \\ wesley.spikes@gmail.com}

\begin{document}

\printdraftwatermark % DRAFT

%----------------------------------------------------------------------------------------
%	SUMMARY
%----------------------------------------------------------------------------------------

\begin{rSection}{Summary}
I like challenges. I'm for an opportunity for growth, I handle ambiguous situations well, and I can break down complex problems into actionable tasks.
% FIXME: Maybe include something about successfully juggling multiple things going on at the same time?
\end{rSection}

%----------------------------------------------------------------------------------------
%	TECHNICAL STRENGTHS SECTION
%----------------------------------------------------------------------------------------

\begin{rSection}{Technical Strengths}

\begin{tabular}{ @{} >{\bfseries}l @{\hspace{6ex}} l }
Computer Languages & Perl, Bash, Python, Ruby, \LaTeX \\
Operating Systems & Linux (Debian, Ubuntu, Gentoo), OS X, Windows \\
Software & Apache, Chef, Nginx, git, vim, MySQL, Jenkins, Gerrit, KVM \\
Concepts & Application Security, CI/CD, Systems Configuration
\end{tabular}

\end{rSection}

%----------------------------------------------------------------------------------------
%	WORK EXPERIENCE SECTION
%----------------------------------------------------------------------------------------

\begin{rSection}{Experience}

\begin{rSubsection}{Personal Break}{Sep 2016 - Present}{}{}
% FIXME: KF recommends something other than "Personal Break" -- maybe "Freelance"
%    "and make the stuff you were doing a bit more formal"
%    "when someone asks, you can go into detail, but at first glance that might scare someone off"
%    "at least, worded that way, it might spook someone"
% My concern with "Freelance" is that it implies paid jobs for other persons/companies, which isn't true.
\item Took a 6 month break, to travel, recharge, and explore LA. Best decision.
\item Learned some C\# and Unity to mentor my youngest brother, who wants to get into game development.
%        FIXME: ^ Maybe remove the family bit. And I don't think a reader cares why the other person wants to learn.
%\item Playing most of the puzzle games in my Steam backlog. (FIXME: Maybe don't include)
%\item Reverse engineering games for fun. After beating a game, it's a fun challenge to reverse it and find little ways to tweak them. (FIXME: Comments below)
%   Example projects: Cookie Clicker (simple/boring) and AdCap.
%   AdCap had a non-standard LZF compression routine (slightly non-standard/buggy, as other lzf libs wouldn't decompress it), as well as writing a custom MITM server to "validate" support codes to give free resources.
%    Not a game, but I also recently RE'd Charles Proxy (which I already lawfully purchased -- I'm only in it for the challenge) and wrote a keygen for it, because why not. The keygen source was immediately put into an encrypted archive volume after it was confirmed working, not to be distributed, because ethics.
\item Set up a personal lab in my office with 4 physical servers.
%   Configured the lab's gateway server as a router w/ DHCP on the lab network
%   Configured TFTP and DHCP server for bootstrapping bare metal machines
\item Reverse engineering software and embedded firmware.
%   Goal is wanting to better understand RE'ing embedded systems
%   Example projects: Sonos Play:1 (current) and Berg Little Printer (past)
%   Sonos Play:1 firmware is hampered by not having a license for a good RE tool like IDA -- Hopper, Binary Ninja, and similar affordable tools don't do a good job of processing binaries for this arch (PowerPC Gen2 a la PowerQUICC II Pro). :\
\end{rSubsection}

%------------------------------------------------

\begin{rSubsection}{DreamHost}{Oct 2014 - Aug 2016}{Senior/Lead Software Engineer}{Los Angeles, CA}
\item Managed two direct reports.
%  FIXME: What about the mentoring I'd done? Worth calling out specifically?
%  One was a Jr Dev, who I'd worked with to get them to leveled up to Dev.
%  The other Jr was on track to level up when I left, and within a few months of my departure, has successfully gone to Dev. (This one probably needs some massaging, since it feels a bit wrong)
%  Beyond simply mentoring in place, encouraging them to reach out to other teams directly and be confidence enough in that regard.
\item Inter-team coordination with Systems Engineering, Security, Technical Support, and Corporate IT.
\item Managed core development infrastructure including Gerrit and Jenkins.
\item Entrusted with various sets of secure credentials. Primary maintainer of deployed credentials.
\item Post-exploit incident response.
\item Modernized deployment of the primary codebase and it's dependencies (fully documented), cutting deployment time down from 20 minutes to around 45 seconds.
\item Technical and architectural lead on the DreamPress 2, VPS-SSD, and Malware Remover projects.
\item Order of Baku - Very active in internal bug bounty program, reporting several notable issues.
%    More than 11 significant security vulnerabilities reported before we Stopped counting, lol.
\end{rSubsection}

%------------------------------------------------

\begin{rSubsection}{DreamHost}{Jan 2011 - Oct 2014}{Software Engineer}{Los Angeles, CA}
\item Configuration management using both Chef and our in-house system.
\item Projects included the launch of mod\_pagespeed (beating competitors to market) and the transition from subversion to git (including documentation and internal support).
% FIXME: Phrasing
% Other projects that I can remember from this period:
%   Dedi Relaunch
%   ETL and report building for billing systems (chart.io)
%   Replicon integration (maybe. that was a nightmare project)
%   mod_pagespeed launch
%     Google launched it with GoDaddy announced as a launch partner. While Google was still "working with GoDaddy" to roll it out it, our team managed to test it and launch it, beating GoDaddy to market.
%     https://www.dreamhost.com/blog/2010/11/04/mod_pagespeed-now-available/
%   Svn->Git transition (from end-to-end -- docs, support, migration itself, etc)
%   LogicBoxes (maybe. that was a nightmare project)
\item Major contributions to the foundations/core of an old codebase (1997), without breaking production.
% FIXME: Improve the language at the end of the sentence, shift towards positive language rather than negative.
%    Proud of this, given the age and the fact that DH doesn't have good testing overall. (FIXME: Phrasing, also see the comments below.)
%   Object->_classdef (deprecating and safely removing v0 and v1 classdefs, with no errors, issues, or breakage)
%   Object->AUTOLOAD (deprecating and safely removing runtime accessor generation from a large amount of the codebase; project was still in progress at the time I left)
%   Nza->AjaxStep (making Ajax support a first-class citizen of the in-house web framework, without breaking existing code -- essentially converting a hacked-in only-sometimes-working bug -- no joke, a bug -- to an actual feature.
%   Nza -- steps can now 'use strict', which wasn't possible before. Code without 'use strict' in Perl land is a pretty massive issue, leading to many a . In the process of migrating existing steps to this new system, we discovered quite a few potential security issues that we were able to fix before they were ever discovered.
%   Controllers / central launcher / servicectl
%   Services -- migrated executable commands to an actual naming prefix to ensure that code isn't inadvertently and inappropriately executed, while still not breaking old code (sc_renamed)
\item Notable contributions to the billing systems, improving it's resilience and extensibility.
\item Throughly documented how the fraud detection systems worked.
%   That billing disaster that took a couple weeks to clean up (INVNUM, or Merchant Invoice Number, being documented as Optional in the PFPro docs, but the library we were using was using it in a rather terribad way.)
\item Researched and documented many obscure settings and flags of the systems, which were depended on without knowing what their function.
\item Fixed many security issues in the codebase.
%\item Audits -- for financial, security, admin, business, whatever
\item Introduced CI processes from end-to-end -- initial research/prototyping, getting buy-in, implementation, training, and finally supporting the new system.
\end{rSubsection}

%------------------------------------------------

\begin{rSubsection}{DreamHost}{Nov 2009 - Jan 2011}{Technical Support}{Los Angeles, CA}
\item In those days, Support was functionally similar to that of Junior Systems Admin.
\item Specialized in tackling problems others avoided, where it was especially difficult to find the root cause.
% For example, when a customer's databases all disappeared on a routine mysqld restart, successfully locating, restoring, and sending a summary on what went wrong.
\item Diagnosing loady systems and troubleshooting customer websites.
\item Configuring systems - Linux-VServer, DNS, Apache, Nginx, users, and PHP.
\item Helping customers with a 99\% satisfaction rate.
\end{rSubsection}

%------------------------------------------------

\begin{rSubsection}{Page Provider}{Dec 2007 - Nov 2009}{Systems Admin}{Rancho Cucamonga, CA}
\item Small, 1-rack hosting provider.
\item Building and deploying custom kernels and software for colocated customers.
\item Provisioning and deploying servers and services.
\item Configuration and management of Apache, users and databases.
\end{rSubsection}

\end{rSection}

%----------------------------------------------------------------------------------------

\end{document}
